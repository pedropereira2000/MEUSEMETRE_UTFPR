
Ética do dever, deontológia e imperativo categórico - Immanuel Kant

Immanuel Kant foi expressivamente influente pelos seus ideais iluministas da época, e com seus fundamentos laico. Sempre tentando levar o pensamento a ser uma faculdade autônoma e livre de amarras, principalmente da igraja medieval da época. Ele acreditava que esse pensamento autônomo poderia também conduzir os indivíduos ao esclarecimento e a maioridade da consciencia racional e chegar sozinho a decisão do que é dever.

Kant vem a descrever ética do dever e a deontologia como sendo racionalmente propostos que o dever é entendido como a finalidade da própria ação, sendo assim quebrando a idéia proposta pela sociedade e pensamentos mais antigos, que assumian como verdade a idéia de que os fim justificam os meio e tendo como finalizadade a ação. E para essas pensamentos e tradições as ações são moralmente relacionadas com um fim determinado como um objetivo das ações humanas.

O imperativo categórico proposto por Kant vem de uma fórmula moral para a resolução de questões ralativas à ações e ao longo das obras de Kant, aparece formulado de três maneiras direntes, sendo essas a 1° - Age como se a máxima de tua ação devesse ser erigida por tua vontade em lei universal da natureza. 2° - Age de tal maneira que trates a humanidade, tanto na tua pessoa como em outras pessoas, sempre como um fim e nunca como um meio. 3° - Age como se a máxima de tua aação devesse servir de lei universal para todos os seres racionais.

Links:
1° - https://www.todamateria.com.br/etica-kant-imperativo-categorico/
2° - https://politicalivre.com.br/artigos/etica-deontologica/

Ética da Compaixão - Arthur Schopenhauer

Arthur Schopenhauer um filósofo alemão que se apossou das idéias de dever e do imperativo categórico de Kant e se volta para as dores do mundo de forma indagativa, não mais platônica ou kantiana, pois agora com Schopenhauer buscando na compaixão fundamento para a ética, renegando o abstrato imperativo categórico que se baseia no dever.

Sobre a ética da compaixão é de suma se destacar que o mundo descrito por Schopenhauer é vontade e representação, onde não conhecemos o mundo em si, mas o mundo que nos é apresentado através dos sentidos e processados por nosso aparelho cognitivo. Não conhecemos o mundo, mas os fenômenos que se apresentam, pois só conhecemos a nossa representação da realizadade mas não a realizadade em si.

Schopenhauer propõe uma ética prática e vivencial baseada na compaixão, apesar do egoísmo e da crueldade que fazem parte da existência humana, a caridade e a compaixão são o contraponto do egoísmo, e este é fruto do eu e do ego, que fazem o homem se considerar o centro do mundo se opondo violentamente contra tudo que impeça seu bem-estar.

A compixão, enquanto princípio fundamental, é proposta na ética de Schopenhauer, contra a razão pura kantianam pois Schopenhauer acredita na compaixão inata como princípio ético pois vai além da nossa representação de mundo, por conta disto estabelece uma metafísica baseada na vontade universal, que é essência de todos os seres, da qual tanto o egoísmo quanto a caridade fazem parte.

Links:
1° - https://www.netmundi.org/filosofia/2015/a-etica-em-shopenhauer/
2° - http://www.ufrrj.br/graduacao/prodocencia/publicacoes/etica-alteridade/artigos/Renato_Nogueira.pdf